\documentclass[report_18month.tex]{subfiles}
\begin{document}
Climate change and air pollution, caused primarily by burning fossil fuels, pose grave risks to public health, infrastructure and productivity. Recognising this, 195 countries have signed the Paris climate accord in 2015, each pledging to reduce their greenhouse gas (GHG) emissions according to their Intended Nationally Defined Contribution (INDC). While ambitious, even the INDCs are not believed to be sufficient to keep the global temperature from exceeding 2$^o$ C above pre-Industrial levels, when catastrophic damage is likely to occur \citep{rogelj2016paris}.

\section{UK Political Context}
\label{sec:UK Political Context}
The commitments the UK has made towards mitigating climate change are given in Table \ref{table:UKcommits}. Reduction targets are quoted relative to a 1990 baseline.
%\begin{center}
\begin{longtable}{ M{2.4cm} M{1.2cm} M{1.2cm} M{4.95cm} M{4.05cm} }
\caption{Commitments relating to energy and climate change, to which the UK is signatory.}\\
\renewcommand*{\arraystretch}{1.25}
%\begin{tabular}{ M{2.4cm} M{1.2cm} M{1.2cm} M{4.5cm} M{4.5cm} }
\textbf{Commitment} & \textbf{Year Signed} & \textbf{Target Year} & \textbf{Terms} & \textbf{Status} \\ 
\hline
Kyoto Protocol & 1997 & 2012 & Reduce GHG emissions by 13 \% \citep{UKccc2008}. & Target met in 2009 \citep{beis2017ghg}. \\
\hline
UK Climate Change Act & 2008 & 2050 & Reduce GHG emissions by 80 \% \citep{UKccc2008}. & Will be missed unless reductions accelerate \citep{beis2017ghg}. \\
\hline
EU targets & 2010 & 2020 & 15 \% of energy must come from renewable sources; including 30 \% of electricity, 12 \% of heat, 10 \% of transport \citep{ccc2016targets2020}. & On track for electricity target; heat and transport targets likely to be missed \citep{ccc2016targets2020}. \\
\hline
Paris Agreement & 2015 & 2030 & The EU must reduce GHG emissions by 40 \%. 27 \% of energy must come from renewable sources. \citep{ccc2016paris} & Progress depends on other EU countries; UK Climate Change Act imposes a stricter target. \\ 
\hline
%\end{tabular}
\label{table:UKcommits}
\end{longtable}
%\end{center}

The technology to reduce fossil fuel usage exists in the form of wind, solar and other renewable power sources. The installed capacity of wind has grown 20-40 \% per year between 2000-2012, while solar photovoltaic (PV) has grown 20-70 \% per year in the same period \citep{carbajales2014storage}. With a learning rate of 4 \% for each doubling of cumulative installed wind capacity, and 13-20 \% for PV, these falling costs give reason for hope, especially for developing countries where a low energy price is vital for industrialisation and improving quality of life \citep{tol2017socialcost}.

It is generally accepted that carbon emissions must be reduced to net zero - and even then some residual planetary warming is unavoidable \citep{frolicher2014co2}. To this end, much research has been done to construct scenarios imagining how various countries can realistically achieve net zero carbon emissions by some date within the next few decades. \citep{sluisveld2016nweu,eu2011roadmap2050,chai2014modeling,ccc2008} These invariably include:
\begin{itemize}
\item Transport powered by electricity (or hydrogen or biofuels) instead of petrol, diesel and other fossil fuels
\item Heating powered similarly (and/or with solar thermal), rather than using natural gas or heating oil
\item Expansion of electrical generation capacity to accommodate electrified heating and transport
\item Replacing fossil-burning power plants with some combination of renewable and nuclear power sources
\item Energy storage to keep electrical supply and demand in instantaneous balance continuously
\end{itemize}

The policies that various UK governments have implemented, and in some cases reversed, attempting to meet the targets in Table \ref{table:UKcommits}, are summarised in Table \ref{table:UKactions}.
%\begin{center}
\begin{longtable}{ M{2.4cm} M{1.3cm} M{7.6cm} M{3cm} }
\caption{Clean energy policies implemented by UK governments 1997 - present day.}\\
\renewcommand*{\arraystretch}{1.25}
%\begin{tabular}{ M{2.4cm} M{1.3cm} M{7.6cm} M{3cm} }
\textbf{Policy} & \textbf{Year Started} & \textbf{Terms} & \textbf{Current Status} \\ 
\hline
Renewables Obligations (RO) & 2002 & Electricity suppliers were obliged to source a given percentage of their supply from renewable sources. Renewable generators were issued with certificates (ROCs) which could be traded on an open market \citep{ofgem2017roc}. & Ended March 2017. \\
\hline
Vehicle Scrappage scheme & May 2009 & Cars older than 10 years could be traded in for a \textsterling2000 discount on a new (more efficient) car, funded half-and-half by government and the manufacturer of the new car \citep{harari2009vehiclescrap}. & Ended February 2010 (after an extension). \\
\hline
Feed-in Tariffs (FiT) & 2010 & Renewable installations would receive a subsidy per kWh generated. The FiT level was reduced each year in anticipation of falling prices, but once signed up, an installation would receive FiT for its whole lifetime fixed at the level it locked on to \citep{ofgem2017fit}. & Ongoing; sudden reduction in December 2016. \\
\hline
Smart meter roll-out & 2011 & All homes and businesses to have a smart meter by 2020 \citep{ofgem2017smartmeters}. & Unlikely to be met. \\
\hline
Green Deal & 2011 & Energy efficiency measures were offered at no up-front cost, with payment taken out of the participant's energy bills \citep{greendeal2017}. & Ended in 2015 due to low take-up. \\
\hline
Capacity Market (CM) & 2014 & Auctions settling payments to standby generators which are needed as backup for intermittent renewable generation \citep{mattholie2017cm}. & Ongoing, with some rule changes. \\
\hline
Contracts for Difference (CfD) & 2014-15 & If the market sale price for electricity falls below an agreed strike price, the difference is paid to the contract holder as a subsidy \citep{beis2017cfd}. & 2nd round held and closed in 2017. \\
\hline
Coal phase-out & 2016 & No coal power stations to be operating by 2025 \citep{govuk2016coal}. & On track, discounting biomass co-firing. \\
\hline
Phase-out of Triad payments to embedded generators & 2017 & Generators have been able to claim Triad payments (charged to consumers during the three half-hour periods of highest demand nationally but separated by at least 10 days, designed to incentivise demand reduction during the whole winter, to relieve transmission constraints). These payments are being phased out for generators below 100 MW, as they contribute little to relieving transmission constraints \citep{ofgem2017triads}. & To be replaced with payments around 10$\times$ smaller, for relieving distribution constraints. \\
\hline
Petrol and diesel phase-out & 2017 & No vehicles running on petrol or diesel to be sold in the UK by 2040 \citep{petroff2017ev}. & Some announcements from car manufacturers that they will comply. \\ 
\hline
%\end{tabular}
\label{table:UKactions}
\end{longtable}
%\end{center}

The plethora of clean energy policies have received criticism for various reasons. Unintended consequences are a recurring theme: many Capacity Market contracts were awarded to diesel generators and old coal plants, effectively subsidising inefficient operation by paying them to remain idle for much of the time and emit more GHGs \citep{mattholie2017cm}. The coal phase-out (or arguably prevailing market conditions) have led coal power stations to switch to co-firing with biomass, whose emissions reduction benefits will take time to accrue, and depend on conditions that are not fulfilled with certainty: the biomass source must be fast-growing, re-planted faster than it is harvested, not diverted from a non-emitting use such as making paper, and of a species aligned with local conservation objectives \citep{lamers2013biomass}. Many of the new cars bought through the vehicle scrappage scheme were diesel vehicles, whose NOx emissions contribute to urban air pollution, a problem now recognised as needing attention simultaneously to climate change \citep{petroff2017ev}.

Another category of complaint is that renewable energy installations were unfairly rewarded by RO, FiT, CfD, and Triad payments pre-phase-out. It can be argued that funding an expensive climate mitigation project over a more cost-effective one results in a sub-optimal allocation of resources to tackle the problem. On the other hand, could the phenomenal decline in the cost of solar PV in the past decade have occurred without such intervention from governments in the UK, Germany, China and elsewhere in the world?

In any case, it is clear that UK energy policy is continually evolving. This can have a de-stabilising effect on immature industries, but also shows a responsiveness to dynamic market conditions and technical requirements. It is important for each renewable energy project to become financially self-sustaining as far as possible, by rendering services of genuine value.

\section{Technical Challenges of Integrating Low-Carbon\\ Non-Despatchable Power Sources}
\label{sec:Technical Challenges of Integrating Low-Carbon Non-Despatchable Power Sources}
These power sources are defined as intermittent renewable sources. Wind and solar PV come within this category, but not biomass, for example, as the power output of a biomass plant can be controlled by its operator. Nuclear power may be counted as non-despatchable, as it is slow and costly to vary from a constant nominal output. In addition to the socio-political barriers to their integration described in Section \ref{sec:UK Political Context}, there are a number of technical challenges. Most of these can be solved by energy storage, which may form the basis of genuinely valuable services that storage can render in order to make renewable energy financially self-sustaining.

\subsection{Reduction of Total System Inertia}
\label{sec:Reduction of Total System Inertia}
The total inertia of all the synchronous generators in a country keeps them spinning for a time on the order of seconds after unexpected events such as sudden failure of a generator, or mis-forecast wind generation or electrical demand \citep{kirby2004frequency}. This gives the system operator time to respond to the event, for example by bringing reserve generation online. As fossil-fuelled generators are replaced with renewable sources connected via power electronic converters, the consequent reduction in total inertia reduces the time available for the system operator to respond to unexpected events. With more intermittent renewable sources, the impact of mis-forecast generation is increased, exacerbating the effect \citep{klimstra2014power}.

Unexpected events cause power imbalance in the transmission system. This leads to deviations in the AC frequency (higher than nominal if supply exceeds demand, and vice versa). The frequency deviation can be sensed and corrected for in a feedback loop, to restore power balance in the transmission system. Currently, this is achieved by power stations ramping their output up (in response to below-nominal frequency) or down (above-nominal frequency), as an ancillary service paid for by the Transmission System Operator (TSO), National Grid \citep{natgrid2013ffr}. If non-despatchable power sources come to dominate, frequency response can no longer be provided solely by power stations, but must come at least partly from demand response or energy storage.

\subsection{Network Upgrade Requirements}
\label{sec:Network Upgrade Requirements}
The electricity network of the UK and most developed nations was not designed to accept power injection at the distribution level, where most wind and solar installations connect. The Distribution Network Operator (DNO) must ensure thermal and voltage limits on the distribution network are not breached, which may require reinforcement of power lines and upgrading of transformer tap changers. \citep{glover2016power} The DNO charges a connection fee to cover these costs. Responses to a consultation held by National Grid indicate connection fees averaging \textsterling130,000/MW \citep{natgrid2013consult}.

Energy storage allows time-shifting of power export from non-despatchable sources, such that most of the energy generated can still be exported to the network while respecting an export power limit below the installed generation capacity. Thus energy storage can save on connection fees, and save time that would otherwise be needed to upgrade distribution network assets before the renewable generator can come online \citep{howison2017sse}.

\subsection{Ramping Requirements}
The use of wind and solar PV places demanding requirements on the ramp rate of other power plants on the system: wind generation can vary on timescales down to seconds, and solar PV similarly due to the passage of clouds \citep{kirby2004frequency}. Further, the evening decrease in PV output coincides with an increase in electrical demand for cooking and entertainment, especially in developed countries \citep{kirby2004frequency}. Even with a modest penetration of 18 GW PV out of 78 GW total generation capacity in California (only 3 GW of which was utility-scale in 2013, the rest being domestic or commercial rooftop) \citep{cec2016california}, this so-called `duck curve' effect leads to a ramping requirement of over 4 GW/hour \citep{denholm2015duck}. This is why expansion of renewable energy is often accompanied by expansion of gas generation plant capacity, and, following the introduction of the Capacity Market in the UK, for example, disastrously polluting diesel plants too \citep{mattholie2017cm}. The ramping function must be provided in a less polluting, less carbon-intensive way, such as by energy storage.

\subsection{Curtailment from Oversizing}
The load factor of intermittent renewable sources tends to be low: 25-35 \% for wind, 10-16 \% for solar PV \citep{carbajales2014storage}, varying by location and time of year. The comparatively high load factor of electrical load (around 60 \% nationally in the UK \citep{gridwatch2017}, also varying across the year) means generation capacity must necessarily be oversized if most or all of the load is to be met by renewable sources (as it must be if net zero carbon emissions are to be achieved). This would lead to curtailment of energy at times when supply exceeds demand, to mitigate the problems described in Section \ref{sec:Reduction of Total System Inertia}. Even with the UK's modest wind capacity of 16 GW \citep{beis2017renewables} (out of a total generation capacity of nearly 80 GW \citep{beis2017cap}), wind farms are controversially already sometimes paid to curtail their generation \citep{blake2017windcurtail}.

By installing energy storage as described in Section \ref{sec:Network Upgrade Requirements}, some curtailment of generation can be avoided. This means more revenue from electricity sales for the renewable generator, less public money spent on payments to curtail generation, and increased utilisation of the resources that were used to manufacture the renewable generation assets.

\section{The Problems with Batteries}
Rechargeable battery storage is chosen as the focus of this work. It is a mature technology, with straightforward integration with solar PV, the fastest-growing type of renewable energy. Furthermore, batteries are usable in any location (unlike pumped hydroelectric storage which is limited to mountainous areas), and their sub-second response time and good balance between power and energy \citep{luo2015overview} make them suitable for the uses described previously in Section \ref{sec:Technical Challenges of Integrating Low-Carbon Non-Despatchable Power Sources}.

However, rechargeable batteries typically do not last beyond a few years \citep{luo2015overview}, making the high initial costs difficult to recoup over the short battery lifetime. \citep{zhang2017sweden,abdulla2017aus,nottrott2013pvbatt,stadler2009expensiveco2} Environmentally, the manufacture and disposal of batteries also poses problems: mining and refining the raw materials, processing them, recycling at the end of battery life, all consume energy and thus lead to emissions that contribute to global warming, acid rain, eutrophication of waterways, and ozone depletion \citep{romare2017life,peters2017aqueous}. Furthermore, the mining of raw materials leads to their depletion, and damages the health and welfare of mine workers \citep{mcmanus2012lithium}. Health risks also burden the workers involved in manufacturing and recycling batteries, especially those conducting informal lead-acid recycling operations in developing countries \citep{haefliger2009leadpoison,gottesfeld2011lead}.

A solution to low profitability is to draw on multiple revenue streams \citep{tsagkou2017stack,moreno2015milp}. An additional way is to use second-life Electric Vehicle (EV) batteries - that is, batteries that no longer have sufficient power density and energy density for mobile applications, but are still usable for stationary applications \citep{gaines2014evbatt,jiao2016evbiz}. As well as being lower cost, second-life batteries would displace some manufacture of new stationary batteries, and their associated environmental and social issues.

Existing publications on service provision by battery systems tend to focus on profit maximisation. A particularly egregious trend is for justifying batteries' green credentials by their ability to increase on-site consumption of renewable electricity \citep{zhang2017sweden,hoppmann2014selfcons,weniger2013sizing}, while ignoring effects on the system outside this artificial boundary. A more appropriate methodology would be to compare alternative scenarios with and without batteries. For example, McKenna \emph{et al.} found that the way existing energy storage is operated on the Irish electricity grid slightly worsens the country's GHG emissions overall \citep{mckenna2017eir}. Significant emissions savings are possible by using energy storage for curtailment avoidance, but this application is not profitable compared to the others they studied.

\section{Summary}
There is growing concern for environmental issues such as climate change and air pollution, as evidenced by the commitments being made in the UK and around the world. The policy landscape of the UK in areas relating to energy and the environment, is dynamic and unstable. Renewable energy projects must be able to sustain themselves financially if there is to be any hope of reaching net zero carbon emissions. The challenges posed by integration of renewable energy into the electricity network can be solved by energy storage such as rechargeable batteries. However, batteries pose further problems due to their expense and the steep social and environmental costs of their manufacture and recycling.

To address these issues, a novel solar PV farm design is proposed. By connecting batteries and PV panels in a so-called `solar+' system, it becomes possible to expand the PV generation capacity beyond the grid export limit, using the battery to store energy that would otherwise be curtailed and then discharging it later when PV output falls below the export limit. Though the use of such a system for selling electricity on the wholesale market has been investigated before \citep{zhang2017sweden}, no detailed work exists yet on combining this function with ancillary service provision to boost revenues, and the implications of using second-life batteries for such an application. Nor does much work exist on the environmental impact of a single grid-connected system, as opposed to an isolated micro-grid \citep{fadaee2012standalone,stadler2009expensiveco2}, or on a national or global scale \citep{strbac2012strategic,bussar2016eu,carbajales2014storage}. But in a free market context, large-scale decarbonisation depends on it being possible and profitable on the scale of individual systems.

By designing, modelling, and optimising the system, calculating its emissions costs and savings, and performing sensitivity analyses, the following questions will be investigated:
\begin{itemize}
\item What is the interplay between maximising profit and saving emissions - a pure trade-off, or does some synergy exist?
\item Can a solar+ system break even without subsidy while still saving significant emissions?
\item How sensitive are the results to details in the modelling, such as battery degradation, and grid emissions intensity?
\item Do the increased conversion losses of less efficient second-life batteries outweigh their benefit in displacing manufacture of new batteries?
\item Should a price be imposed on GHG and pollutant emissions, and how much?
\end{itemize}

It is expected that the findings will be of interest firstly to researchers developing battery, PV, and power converter technologies, to build a case for focusing research efforts in particular directions depending on the objectives - profit maximisation and emissions reduction. Secondly, to investors and consultants whose decisions benefit from a firm knowledge base, to be contributed to by this project. And finally, to policy-makers who are tasked with designing the incentives and market frameworks to achieve emissions reductions with a minimum of unintended consequences.

The remainder of the thesis is organised thus: Chapter \ref{sec:The Value Proposition} reviews existing literature on `solar+' value propositions, leading to the development of the proposed design and operating strategy. Chapter \ref{sec:Calculation Methodology} describes how the system's Net Present Value (NPV, that is, whole-life profit) will be calculated, including a review of inputs and assumptions. Chapter \ref{sec:Predicting the Price of Second-Life Electric Vehicle Batteries} presents a novel method for estimating the long-term price evolution of second-life batteries, which is used in financial analysis of the system. Chapter \ref{sec:Projections of Grid Emissions Intensities} presents work on producing improved high-resolution projections of grid emissions marginal intensity, for more accurate accounting of the environmental benefits of the system in displacing power generation elsewhere on the grid. Chapter \ref{sec:Modelling Battery Degradation} details the method for incorporating battery degradation in the system analysis. Chapter \ref{sec:Calculating Connection Charges for Alternative Systems} reviews the financial and environmental costs of alternative ways of expanding solar generation capacity. These inputs are all brought together in Chapter \ref{sec:Sizing the System}, which presents the results of optimising the system component sizes, and analyses their sensitivity to the different inputs. Chapter \ref{sec:Discussion} discusses these results, their implications, and suggests further work. The conclusion is Chapter \ref{sec:Conclusions}.
\end{document}
